


\documentclass[11pt]{article}

\usepackage{amsmath}
\usepackage{indentfirst}
\usepackage{natbib}

\RequirePackage{amsfonts}
\newcommand{\real}{\mathbb{R}}

\begin{document}

\title{Trust Regions}
\author{Charles J. Geyer}
\maketitle

\section{Trust Region Theory}

We follow \citet[Chapter~4]{naw}, using their notation.
\citet[Section~5.1]{flet} discusses the same algorithm,
but with slight differences.

\subsection{Main Loop}

Suppose $f : \real^n \to \real$ is a function we wish to minimize,
and we do so by producing a sequence $x_1$, $x_2$, $\ldots$ of points
in $\real^n$ that converge to a solution.

This sequence of iterates is produced by the trust region main loop
which repeatedly solves the \emph{trust region subproblem}.  First we form the
Taylor series expansion of $f$ about $x_k$
$$
   m_k(p) = f_k + g_k^T p + \tfrac{1}{2} p^T B_k p
$$
where
\begin{align*}
   f_k & = f(x_k)
   \\
   g_k & = \nabla f(x_k)
   \\
   B_k & = \nabla^2 f(x_k)
\end{align*}
and the idea is we ``trust'' the Taylor series expansion only so far,
so we minimize $m_k$ subject to the constraint $\lVert p \rVert \le \Delta_k$
where $\lVert \,\cdot\, \rVert$ denotes the Euclidean norm.

Suppose $p_k$ is a solution to the trust region subproblem
(about which more in Section~\ref{sec:subproblem} below).
The adjustment of $\Delta_k$ is done as follows
\citep[Algorithm~4.1]{naw} based on
\begin{equation} \label{eq:rhok}
   \rho_k = \frac{f(x_k) - f(x_k + p_k)}{m_k(0) - m_k(p_k)}
\end{equation}
which is the actual decrease in the objective function $f$ in the step
compared to the predicted decrease using $m_k$.  If $\rho_k$ is small
or negative, then $m_k$ is a bad model at $x_k + p_k$ so the step should
not be used and the trust region radius should be adjusted.
The complete trust region main loop body solves the trust region subproblem
and then adjusts the current point and trust region radius as follows
\begin{tabbing}
\textbf{if} ($\rho_k < 1 / 4$) \textbf{then} \\
\qquad \= $x_{k + 1} := x_k$ \\
\> $\Delta_{k + 1} = \lVert p_k \rVert / 4$ \\
\textbf{else} \\
\> $x_{k + 1} := x_k + p_k$ \\
\> \textbf{if} ($\rho_k > 3 / 4$ and $\lVert p_k \rVert = \Delta_k$)
\textbf{then} \\
\> \qquad $\Delta_{k + 1} = \min( 2 \Delta_k, \bar{\Delta} )$ \\
\> \textbf{fi} \\
\textbf{fi}
\end{tabbing}
where $\bar{\Delta}$ is the maximum allowed $\Delta_k$.

\subsection{Trust Region Subproblem} \label{sec:subproblem}

Now we follow Section~4.2 of \citet{naw}.  We drop subscripts
writing $m$ instead of $m_k$ and so forth.  So the subproblem is
\begin{align*}
   \text{minimize} &
   \quad m(p) \stackrel{\text{def}}{=} f + g^T p + \tfrac{1}{2} p^T B p
   \\
   \text{subject to} & \quad \lVert p \lVert \le \Delta
\end{align*}
which minimizes a quadratic function over the closed ball of
radius $\Delta$ centered at the origin.

A vector $p^*$ is a global solution to the trust region subproblem
if and only if
\begin{subequations}
\begin{equation} \label{eq:primal}
   \lVert p^* \rVert \le \Delta
\end{equation}
and there exists a scalar $\lambda \ge 0$ such that
\begin{gather}
   (B + \lambda I) p^* = - g
   \label{eq:derivative}
   \\
   \lambda = 0 \quad \text{or} \quad \lVert p^* \rVert = \Delta
   \label{eq:complementary}
   \\
   B + \lambda I \ \text{is positive semidefinite}
   \label{eq:second-order}
\end{gather}
\end{subequations}
\cite[Theorem~4.3]{naw}.

\subsubsection{Unconstrained Solutions}

The $\lambda = 0$ case is easy.  If $B$ is positive definite and
$p^* = - B^{-1} g$ satisfies \eqref{eq:primal}, then that is the solution.

\subsubsection{Constrained Solutions}

The $\lVert p^* \rVert = \Delta$ case is more complicated.  Define
\begin{equation} \label{eq:plambda}
   p(\lambda)
   =
   - (B + \lambda I)^{-1} g
   =
   - \sum_{j = 1}^{n} \frac{q_j^T g}{\lambda_j + \lambda} q_j
\end{equation}
where $\lambda_j$ are the eigenvalues of $B$ and $q_j$ are the corresponding
orthonormal eigenvectors (this is valid only when $\lambda \neq - \lambda_j$
for all $j$).  Then
\begin{equation} \label{eq:plambda-norm-squared}
   \lVert p(\lambda) \rVert^2
   =
   \sum_{j = 1}^{n} \left( \frac{q_j^T g}{\lambda_j + \lambda} \right)^2
\end{equation}
The analysis again splits into several cases.

\paragraph{The Easy Case}

Let $\lambda_{\text{min}}$ denote the minimum eigenvalue of $B$.
If
\begin{equation} \label{eq:easy-condition}
   (q_j^T g) \neq 0, \qquad \text{for some $j$ such that
   $\lambda_j = \lambda_{\text{min}}$}
\end{equation}
then $p(\lambda)$ is a continuous, strictly decreasing function on
the open interval $(- \lambda_{\text{min}}, \infty)$ and goes to $\infty$
as $\lambda \to - \lambda_{\text{min}}$ and to zero as $\lambda \to \infty$.
Thus a $\lambda^*$
such that $\lVert p(\lambda^*) \rVert^2 = \Delta^2$ exists and is unique,
equation \eqref{eq:plambda-norm-squared} can be used to find it,
and $p(\lambda^*)$ is the
solution to the trust region subproblem.

\paragraph{The Hard Case}

In the other case, when \eqref{eq:easy-condition} is false,
\eqref{eq:plambda-norm-squared} is continuous
at $\lambda = \lambda_{\text{min}}$ and now 
defines a continuous
strictly decreasing function on
the closed interval $[- \lambda_{\text{min}}, \infty)$, and $\lambda$
must be in this interval in order for \eqref{eq:second-order} to hold.

Now the analysis splits into two subcases yet again.

\paragraph{Hard Case: Easy Part}

If \eqref{eq:easy-condition} is false but
\begin{equation} \label{eq:hard-easy-condition}
   \lVert p(- \lambda_{\text{min}}) \rVert^2 > \Delta^2,
\end{equation}
then there is still a unique $\lambda^*$ in $(- \lambda_{\text{min}}, \infty)$
such that $\lVert p(\lambda^*) \rVert^2 = \Delta^2$,
equation \eqref{eq:plambda-norm-squared} can be used to find it,
and $p(\lambda^*)$ is the
solution to the trust region subproblem.

\paragraph{Hard Case: Hard Part}

Otherwise, when \eqref{eq:easy-condition} and \eqref{eq:hard-easy-condition}
are both false,
we must have $\lambda = - \lambda_{\text{min}}$.
Then $B + \lambda I$ is singular, and \eqref{eq:plambda}
can no longer be used.  Now solutions of \eqref{eq:derivative}
are non-unique, and we have
\begin{equation} \label{eq:ptau}
   p_{\text{hard}}(\tau)
   =
   - \sum_{\substack{j = 1 \\ \lambda_j \neq \lambda_{\text{min}}}}^n
   \frac{q_j^T g}{\lambda_j - \lambda_{\text{min}}} q_j
   +
   \sum_{\substack{j = 1 \\ \lambda_j = \lambda_{\text{min}}}}^n
   \tau_j q_j
\end{equation}
is a solution of \eqref{eq:derivative} for every vector $\tau$.
Since the first sum on the right hand side of \eqref{eq:ptau} has
norm less than $\Delta$ by the falsity of \eqref{eq:hard-easy-condition},
we can choose $\tau$ to make $\lVert p_{\text{hard}}(\tau) \rVert = \Delta$.

Note that the solution $p^*$ obtained in this case is non-unique.
Even if there is only one term in the second sum in \eqref{eq:ptau},
$\tau_j$ of opposite signs produce $p_{\text{hard}}(\tau)$ of the same
length.
When there is more than one term in the second sum in \eqref{eq:ptau},
there are even more possibilities of nonuniqueness.
This nonuniqueness is not an issue, because (at least as far as the subproblem
is concerned) one solution is just as good as another.

\subsubsection{Numerical Stability}

We need to examine how all this case splitting works when the
arithmetic is inexact (as it is in a computer).  Let us take
the eigendecomposition of $B$ that gives us equation \eqref{eq:plambda}
or equation \eqref{eq:ptau}
as authoritative.  We know the eigendecomposition is inexact,
but we do not attempt to
correct for its inexactness.

If all of the $\lambda_j$ as calculated (inexactly) by the eigendecomposition
routine are strictly positive, then \eqref{eq:plambda} with $\lambda = 0$
gives an ``unconstrained'' solution, and this solution has squared norm
\eqref{eq:plambda-norm-squared} that
is either greater than $\Delta^2$ or not.  If not, we have found an inexact
solution.  If greater, we decide to impose the constraint.
This decision is stable in the sense that we will not want to undo it later.

With inexact arithmetic, we are unlikely ever to get the ``hard'' case.
Nevertheless, we allow for the possibility.
Define
\begin{subequations}
\begin{align}
    C_1 & = 
   \sum_{\substack{j = 1 \\ \lambda_j \neq \lambda_{\text{min}}}}^n
   \left( \frac{q_j^T g}{\lambda_j - \lambda_{\text{min}}} \right)^2
   \label{eq:c1}
   \\
    C_2 & = 
   \sum_{\substack{j = 1 \\ \lambda_j = \lambda_{\text{min}}}}^n
   \left( q_j^T g \right)^2
   \label{eq:c2}
\end{align}
\end{subequations}
Then the constrained cases are distinguished as follows.
\begin{itemize}
\item \emph{easy case:} $C_2 \neq 0$.
\item \emph{hard-easy case:} $C_2 = 0$ and $C_1 > \Delta^2$.
\item \emph{hard-hard case:} $C_2 = 0$ and $C_1 \le \Delta^2$.
\end{itemize}

The ``hard-hard case'' is now obvious.  $\tau$ is adjusted so that
the second term on the right hand side in \eqref{eq:ptau} has squared length
$\Delta^2 - C_1$.

In the other two cases we must find a zero of the function of $\lambda$
given by
\begin{subequations}
\begin{equation} \label{eq:phifoo}
   \lVert p(\lambda) \rVert^2 - \Delta
\end{equation}
or (what is equivalent) a zero of the function defined by
\begin{equation} \label{eq:phi}
   \phi(\lambda) = \frac{1}{\lVert p(\lambda) \rVert} - \frac{1}{\Delta}
\end{equation}
which is better behaved \citep[Chapter~4]{naw}.
\end{subequations}
Both are monotone, \eqref{eq:phifoo} strictly decreasing
and \eqref{eq:phi} strictly increasing, but \eqref{eq:phi} is also
nearly linear.

We would like to bracket the zero, finding an interval containing the
zero.  We know that $\lambda = - \lambda_{\text{min}}$ is a lower bound.
We have
$$
   \phi(- \lambda_{\text{min}}) = \frac{1}{\sqrt{C_1}} - \frac{1}{\Delta}
   < 0
$$
in the ``hard-easy case'' and
$$
   \phi(- \lambda_{\text{min}}) = - \frac{1}{\Delta}
$$
in the ``easy case''.  A better lower bound uses
$$
   \lVert p(\lambda) \rVert^2
   \ge \frac{C_2}{(\lambda_{\text{min}} + \lambda)^2}
$$
from which (setting the right hand side equal to $\Delta^2$ and solving
for $\lambda$) we get
$$
   \lambda_{\text{dn}} = \frac{\sqrt{C_2}}{\Delta} - \lambda_{\text{min}}
$$
as a lower bound for the $\lambda$ that
makes $\lVert p(\lambda) \rVert = \Delta$.

To get an upper bound, we note that
if we define
\begin{equation} \label{eq:c3}
    C_3 = 
   \sum_{j = 1}^n \left( q_j^T g \right)^2
\end{equation}
then we have
$$
   \lVert p(\lambda) \rVert^2
   \le
   \frac{C_3}{(\lambda_{\text{min}} + \lambda)^2}
$$
Setting the right hand side equal to $\Delta^2$ and solving for $\lambda$ gives
$$
   \lambda_{\text{up}} = \frac{\sqrt{C_3}}{\Delta} - \lambda_{\text{min}}
$$
as an upper bound.
So now we know there is a solution
in $[\lambda_{\text{dn}}, \lambda_{\text{up}}]$.

\subsection{Rescaling} \label{sec:subproblem-modified}

When the variables are ill-scaled, the trust region is badly designed
and \citet[Section~4.4]{naw} recommend changing the trust region
subproblem to
\begin{equation} \label{eq:problem-rescale}
\begin{split}
   \text{minimize} &
   \quad m(p) \stackrel{\text{def}}{=} f + g^T p + \tfrac{1}{2} p^T B p
   \\
   \text{subject to} & \quad \lVert D p \lVert \le \Delta
\end{split}
\end{equation}
where $D$ is a strictly positive definite diagonal matrix
(as will be seen below, $D$ can be any invertible matrix).

We claim that this is equivalent to running our original algorithm
(with no $D$ or, what is equivalent, with $D$ the identity matrix)
on the function $\tilde{f}$ defined by
$$
   \tilde{f}(\tilde{x}) = f(D^{-1} \tilde{x})
$$
Consider a point $\tilde{x}_k = D x_k$.  Here, if $g$ and $B$ denote
the gradient and Hessian of $f$, then
\begin{subequations}
\begin{align}
   \tilde{g} & = D^{-1} g
   \label{eq:modified-gradient}
   \\
   \widetilde{B} & = D^{-1} B D^{-1}
   \label{eq:modified-hessian}
\end{align}
\end{subequations}
are the gradient and Hessian of $f$.  The trust region subproblem
(of the original kind with no $D$)
for $\tilde{f}$ (for an iterate centered at $\tilde{x}_k$ using
trust region ``radius'' $\Delta_k$) is
\begin{equation} \label{eq:problem-modified}
\begin{split}
   \text{minimize} &
   \quad \widetilde{m}(w) \stackrel{\text{def}}{=}
   f + \tilde{g}^T w + \tfrac{1}{2} w^T \widetilde{B} w
   \\
   & \hphantom{\quad \widetilde{m}(w) \stackrel{\text{def}}{=} \vphantom{f}}
   f + g^T D^{-1} w + \tfrac{1}{2} w^T D^{-1} B D^{-1} w
   \\
   \text{subject to} & \quad \lVert w \lVert \le \Delta_k
\end{split}
\end{equation}
Let $w^*$ be a solution to \eqref{eq:problem-modified}, and
define $p^* = D^{-1} w^*$.  Then $p^*$ solves \eqref{eq:problem-rescale}.

Let $x_1$ is the starting point for the trust region problem
of minimizing $f$ using rescaling $D$.
Define $\tilde{x}_1 = D^{-1} x_1$, and consider it
the starting point for the trust region problem
of minimizing $\tilde{f}$ using no rescaling.
Let $\tilde{x}_1$, $\tilde{x}_2$, $\ldots$ be the sequence of iterates
produced in the latter problem.  Then we have
\begin{align*}
   \tilde{x}_{k + 1} - \tilde{x}_k
   & =
   w^*_k
   \\
   & =
   D p^*_k
   \\
   & =
   D (x_{k + 1} - x_k)
\end{align*}
whenever we have an accepted step (so $x_{k + 1} \neq x_k$).

\section{Termination Criteria}

Although not part of the trust region algorithm proper, termination
criteria are important.  \citet{flet} discusses them on pp.~22--23 and~57 ff.
The conventional criteria are to terminate when any of
\begin{subequations}
\begin{gather}
   f(x_k) - f(x_{k + 1})
   \label{eq:term-on-f}
   \\
   x_k - x_{k + 1}
   \label{eq:term-on-x}
   \\
   \nabla f(x_k)
   \label{eq:term-on-g}
\end{gather}
are small.  All are used in practice.
Note that \eqref{eq:term-on-x} and \eqref{eq:term-on-g} are vectors,
so what ``small'' means for them is more complicated.

\citet[pp.~57 ff.]{flet} makes the point that it is desirable that
an optimization algorithm be invariant under rescaling, perhaps under
arbitrary affine transformation of the domain of the objective function,
perhaps only under diagonal scaling.  A trust region algorithm is not
invariant under rescaling unless the scaling matrix $D$ introduced
in Section~\ref{sec:subproblem-modified} exactly matches the scaling.
Nevertheless, invariant convergence tests still make sense and only
\eqref{eq:term-on-f} among those considered above is invariant.

Fletcher also suggests
\begin{gather}
   (x_{k + 1} - x_k)^T
   \nabla f(x^k)
   \label{eq:term-on-i}
\end{gather}
which is invariant when the step is a Newton step.

It is also possible for $\Delta_k \to 0$ as $k \to \infty$
\citep[see][proof of Theorem~5.1.1]{flet}.  Thus it is also necessary
to consider termination when
\begin{gather}
   \Delta_k
   \label{eq:term-on-d}
\end{gather}
\end{subequations}
is small.

\section{Algorithm}

\subsection{Order of Operations}

The algorithm has one main loop and (as described) does
\begin{itemize}
\item one objective function evaluation (value, gradient, and Hessian) and
\item one adjustment of $\Delta_k$ and $x_k$
\end{itemize}
per iteration.  The main tricky bit is to decide where in the loop
each part of the computation occurs.  We can break the body of the
loop into the following parts.
\begin{itemize}
\item Solve trust region subproblem.
\item Evaluate objective function and derivatives at $x_k + p_k$.
\item Adjust current point $x_k$ and trust radius $\Delta_k$.
\item Test for termination with $x_{k + 1}$ the solution.
\end{itemize}
The following considerations influence the order in which these
steps are done.
\begin{itemize}
\item Evaluation of $f(x_k + p_k)$ must come between the solution
    of the subproblem (which produces $p_k$) and the adjustment of
    $x_k$ and $\Delta_k$, because the adjustment depends on \eqref{eq:rhok},
    which depends on $p_k$.
\item Thus all that remains to be decided is where in the Solve-Evaluate-Adjust
    cycle to put the termination test.
\begin{itemize}
\item Solve-Test-Evaluate-Adjust and Solve-Evaluate-Test-Adjust are
    senseless,
    because we can't decide that $x_k + p_k$ is the solution until after we
    decide whether or not to set $x_{k + 1} = x_k + p_k$ in the Adjust step,
    and it makes no sense to do the work of producing $p_k$ and not let
    $x_k + p_k$ be considered for the solution.
\item Thus the order should be Solve-Evaluate-Adjust-Test as in the list
    above.
\end{itemize}
\end{itemize}

\subsection{Termination Test}

Criteria for termination.
\begin{enumerate}
\item[(a)] The change in the objective function
    value $\lvert f(x_k) - f(x_{k + 1}) \rvert$ is less than
    the termination criterion $\epsilon_1$ in any step.
\item[(b)] The change in the second-order Taylor-series model
    of the objective function
    value $\bigl\lvert (x_k - x_{k + 1})^T \bigl( g_k +
    \tfrac{1}{2} B_k (x_k - x_{k - 1}) \bigr) \bigr\rvert$
    is less than the termination criterion $\epsilon_3$ in
    any step.
\item[(c)] The trust region radius $\Delta_k$ has shrunk to less than
    the termination criterion $\epsilon_4$ in any step.
\end{enumerate}

Condition (b) is also invariant under affine transformations of the
parameter space.  It seems to make more sense than using the first-order
series condition \eqref{eq:term-on-i} discussed above, especially since
we are already calculating it, because it is the denominator
in \eqref{eq:rhok}.

Condition (c) appears to be redundant, since when the trust region
shrinks to too small, this will force (a) and (b) to be small too.

Now we have a nice unification of ideas.
The object tested in (a) is the numerator in \eqref{eq:rhok},
the object tested in (b) is the denominator in \eqref{eq:rhok}.
When either is small, the ratio \eqref{eq:rhok} may be wildly
erroneous due to inexact computer arithmetic, but then we terminate
the algorithm anyway.

However, this analysis tells us that our separation of work to do into
``Adjust'' and ``Test'' steps is erroneous.  When we should terminate,
we cannot adjust because we cannot (or should not) calculate the
``Adjust'' criterion \eqref{eq:rhok}.
The reason why we originally wanted ``Test'' after ``Adjust'' is that
we thought that it made no sense to terminate when the ``Adjust''
procedure rejects the step.
But now we see that if it rejects the step based on a value of
\eqref{eq:rhok} that is reliable, then we won't terminate anyway
because we have decided to terminate if and only if we declare
\eqref{eq:rhok} unreliable.

\begin{thebibliography}{}

\bibitem[Fletcher(1987)]{flet}
Fletcher, R. (1987).
\newblock \emph{Practical Methods of Optimization}, second edition.
\newblock John Wiley.

\bibitem[Nocedal and Wright(1999)]{naw}
Nocedal, J. and Wright, S.~J. (1999).
\newblock \emph{Numerical Optimization}.
\newblock Springer-Verlag.

\end{thebibliography}



\begin{center} \LARGE REVISED DOWN TO HERE \end{center}



\end{document}